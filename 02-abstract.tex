\chap{Abstract} 
Contemporary neural networks, despite their remarkable achievements, often fall short of the robust compositional generalization that characterizes human cognition, particularly in tasks demanding symbolic manipulation and algorithmic reasoning. This dissertation investigates the mechanisms underlying compositional generalization in neural networks and proposes novel neurosymbolic architectures that bridge the gap between connectionist and symbolic computation, primarily by leveraging Tensor Product Representations (TPRs) to embed symbolic structures within vector spaces.

First, I introduce the Role Learning Network (ROLE), a diagnostic model that automatically discovers latent structure in neural representations. This analysis reveals how networks can solve compositional tasks by converging on solutions that approximate compositional vector embeddings of symbolic structures. The causal importance of these discovered structures is demonstrated through activation patching, enabling targeted control over model behavior.

Next, I present the Differentiable Tree Machine (DTM), a unified neurosymbolic architecture that implements symbolic tree operations via a differentiable interpreter. An agent learns to produce a neurosymbolic program, while this interpreter executes the programs. To scale this approach, I develop Sparse Coordinate Trees, a TPR-equivalent encoding scheme that  reduces parameters by 70x, memory by 100x, and latency by 34x. Across a range of distributional shifts from training to testing, DTM with Sparse Coordinate Trees achieves the best out-of-distribution performance compared to both neural and neurosymbolic baselines.

Finally, I focus on enhancing Transformers for modeling formal languages. I analyze the trade-off between parallelism and generalization in Recurrent Transformers for modeling Regular Languages, identifying token-layer recurrence as a key factor and examining how chunk size affects both parallelizability and length generalization. Additionally, I explore augmenting Transformers with stack-like structures for context-free languages, demonstrating that the choice of stack encoding mechanism can significantly impact performance, especially on nondeterministic languages.

Collectively, this dissertation contributes novel analysis techniques (ROLE) and unified neurosymbolic architectures (DTM, sDTM) that integrate differentiable symbolic operations and structured representations within neural networks. By exploring latent structures, explicit tree manipulation, efficient sparse representations, and recurrence, this work offers insights and methodologies for developing next-generation neural models capable of more human-like compositional generalization.

%%%% your abstract goes here (word limit: 350)

%%%%  committee members (add it right after the abstract w/o page break)
\begin{singlespace}

    %% if you have co-advisor, add here w/ \vspace{0.1in} as shown below
    %% alternatively you can use minipage environment to put side-by-side
    \section*{Primary readers}
    
    Dr. Paul Smolensky \\
    Johns Hopkins University, Baltimore MD 

    \vspace{0.1in}

    Dr. Benjamin Van Durme \\
    Johns Hopkins University, Baltimore MD

    \vspace{0.1in}

    Dr. John T. Hale \\
    Johns Hopkins University, Baltimore MD 

    \vspace{0.1in}

    Dr. Colin Wilson \\
    Johns Hopkins University, Baltimore MD

    \vspace{0.1in}

    Dr. Robert Frank \\
    Yale University, New Haven CT 

    \section*{Alternate readers}
    
    Dr. Jennifer Hu\\
    Johns Hopkins University, Baltimore, MD 
    
    \vspace{0.1in}
    
    Dr. Christopher Honey \\
    Johns Hopkins University, Baltimore, MD 

    %% you can add more readers if you have them on your committee 
    %% use \vspace{0.1in} in between members for clarity
    %% you can also place committee members side-by-side using `minipage`


\end{singlespace}